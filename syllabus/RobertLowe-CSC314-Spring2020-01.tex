% Don't touch this %%%%%%%%%%%%%%%%%%%%%%%%%%%%%%%%%%%%%%%%%%%
\documentclass[11pt]{article}
\usepackage{fullpage}
\usepackage[left=1in,top=1in,right=1in,bottom=1in,headheight=3ex,headsep=3ex]{geometry}
\usepackage{graphicx}
\usepackage{float}
\usepackage{longtable}

\newcommand{\blankline}{\quad\pagebreak[2]}
%%%%%%%%%%%%%%%%%%%%%%%%%%%%%%%%%%%%%%%%%%%%%%%%%%%%%%%%%%%%%%

% Modify Course title, instructor name, semester here %%%%%%%%
\title{CSC314: Data Mining}
\author{Dr. Robert Lowe}
\date{Spring, 2020}


% Don't touch this %%%%%%%%%%%%%%%%%%%%%%%%%%%%%%%%%%%%%%%%%%%
\usepackage[sc]{mathpazo}
\linespread{1.05} % Palatino needs more leading (space between lines)
\usepackage[T1]{fontenc}
\usepackage[mmddyyyy]{datetime}% http://ctan.org/pkg/datetime
\usepackage{advdate}% http://ctan.org/pkg/advdate
\newdateformat{syldate}{\twodigit{\THEMONTH}/\twodigit{\THEDAY}}
\newsavebox{\MONDAY}\savebox{\MONDAY}{Mon}% Mon
\newcommand{\week}[1]{%
%  \cleardate{mydate}% Clear date
% \newdate{mydate}{\the\day}{\the\month}{\the\year}% Store date
  \paragraph*{\kern-2ex\quad #1, \syldate{\today} - \AdvanceDate[4]\syldate{\today}:}% Set heading  \quad #1
%  \setbox1=\hbox{\shortdayofweekname{\getdateday{mydate}}{\getdatemonth{mydate}}{\getdateyear{mydate}}}%
  \ifdim\wd1=\wd\MONDAY
    \AdvanceDate[7]
  \else
    \AdvanceDate[7]
  \fi%
}
\usepackage{setspace}
\usepackage{multicol}
%\usepackage{indentfirst}
\usepackage{fancyhdr,lastpage}
\usepackage{url}
\pagestyle{fancy}
\usepackage{hyperref}
\usepackage{lastpage}
\usepackage{amsmath}
\usepackage{layout}

\lhead{}
\chead{}
%%%%%%%%%%%%%%%%%%%%%%%%%%%%%%%%%%%%%%%%%%%%%%%%%%%%%%%%%%%%%%

% Modify header here %%%%%%%%%%%%%%%%%%%%%%%%%%%%%%%%%%%%%%%%%
\rhead{\footnotesize CSC3140-01 Spring 2020}

%%%%%%%%%%%%%%%%%%%%%%%%%%%%%%%%%%%%%%%%%%%%%%%%%%%%%%%%%%%%%%
% Don't touch this %%%%%%%%%%%%%%%%%%%%%%%%%%%%%%%%%%%%%%%%%%%
\lfoot{}
\cfoot{\small \thepage/\pageref*{LastPage}}
\rfoot{}

\usepackage{array, xcolor}
\usepackage{color,hyperref}
\definecolor{clemsonorange}{HTML}{EA6A20}
\hypersetup{colorlinks,breaklinks,linkcolor=clemsonorange,urlcolor=clemsonorange,anchorcolor=clemsonorange,citecolor=black}

\begin{document}

\maketitle

\blankline

\begin{tabular*}{.93\textwidth}{@{\extracolsep{\fill}}lr}
%%%%%%%%%%%%%%%%%%%%%%%%%%%%%%%%%%%%%%%%%%%%%%%%%%%%%%%%%%%%%%

% Modify information %%%%%%%%%%%%%%%%%%%%%%%%%%%%%%%%%%%%%%%%%
E-mail: \texttt{robert.lowe@maryvillecollege.edu} & Office Phone: 865-981-8169 \\

 Office Hours: MWF 1:00PM -- 2:00PM, TR 3:00PM -- 4:00PM  &  Class Hours: TR 11:00 -- 12:15\\
 Office: SSC 214 & Class Room: SSC 204\\
\hline
\end{tabular*}

\vspace{5 mm}

% First Section %%%%%%%%%%%%%%%%%%%%%%%%%%%%%%%%%%%%%%%%%%%%
\section*{Course Description}
Data mining is concerned with the extraction of information from large
amounts of data. This project-based course introduces the concepts,
issues, tasks and techniques of data mining. Topics include data
preparation and feature selection, classification, clustering,
evaluation and validation, and data mining applications. 

% Second Section %%%%%%%%%%%%%%%%%%%%%%%%%%%%%%%%%%%%%%%%%%%
\section*{Required Materials}
\begin{itemize}
    \item \textit{Data Mining and Analysis -- Fundamental Concepts and
    Algorithms} by  Mohammed J. Zaki and  Wagner Meira, Jr.
    \newline\url{http://www.dataminingbook.info}
    \item A shell account on cs.maryvillecollege.edu (alternatively,
        you could install R-Studio locally on your own computer). 
\end{itemize}

% Third Section %%%%%%%%%%%%%%%%%%%%%%%%%%%%%%%%%%%%%%%%%%%%
\section*{Prerequisites}
CSC313 and MTH321 


% Fourth Section %%%%%%%%%%%%%%%%%%%%%%%%%%%%%%%%%%%%%%%%%%%
\section*{Course Goals}
\begin{itemize}
\item Learn and overcome the problems in working with real world data.
\item Learn the difference between good and bad data mining practices.
\item Explore feature extraction and dimension reduction.
\item Use statistical models and machine learning algorithms to interpret data.
\item Learn the R programming language.
\end{itemize}

% Fifth Section %%%%%%%%%%%%%%%%%%%%%%%%%%%%%%%%%%%%%%%%%%%
\section*{Course Structure}
\subsection*{Methods of Instruction}
\begin{itemize}
    \item Lecture
    \item Homework
    \item Projects
\end{itemize}

\subsection*{Grading}
This course is graded using a weighted average among four categories.
The assignments within each category are equally weighted and are all
graded out of 100 points.  Hence your final numeric grade is
computed by finding the average of each category, and then multiplying
them by the corresponding weight.  The weights for each category are
as follows:

\begin{tabular}{|l|r|}
\hline
{\bf Category} & {\bf Weight}\\
\hline
Homework & 40\%\\
\hline
Projects & 60\%\\
\hline
\end{tabular}

\vspace{0.25in}

Letter grades will be assigned according to the following scale:

\begin{tabular}{|lr|lr|lr|lr|lr|}
    \hline
    A+ & 96.7--100\% & B+ & 86.7--90\% & C+ & 76.7--80\% & D+ & 66.7--70\% & F & less than 60 \% \\
    A  & 93.3--96.7\% & B & 83.3--86.7\% & C & 73.3--76.7\% & D & 63.3--66.7\% & & \\
    A- & 90--93.3\% & B- & 80--83.3\% & C- & 70--73.3\% & D- & 60--63.3\% & & \\
    \hline
\end{tabular}

\subsection*{Assessments}
The standards of assessment in each grading category will be as
follows.

\subsubsection*{Homework {\em (40\% of the final grade)}}
Homework will be assigned at various points throughout the semester.
Most of the homework assignments will involve producing documents
using R notebooks.  You will turn in printouts of these notebooks.

\subsubsection*{Projects {\em (60\% of the final grade)}}
There will be three course projects.  These will be completed in
groups of 3-4 students, and each will result in a written report and
a presentation.  Your group will receive a collective grade on the
projects.  Each project is weighted equally.

Note that while your group may elect to have a single presenter on
presentation days, the entire group must be present for the
presentation.  Anyone who does not attend on a presentation day will
receive a reduced grade on the project.  



% Course Schedule %%%%%%%%%%%%%%%%%%%%%%%%%%%%%%%%%%%%%%%%%%%
\subsection*{Schedule}
This is the tentative schedule for our course.  There may be some
slight modifications to the following according to the needs of the
semester.  Presentation days are boxed in on the calendars below.

\begin{tabular}{rrrrrrr}
Su & Mo & Tu & We & Th & Fr & Sa\\
   &    &    &  1 &  2 &  3 &  4\\ 
 5 &  6 &  7 &  8 &  9 & 10 & 11\\ 
12 & 13 & 14 & 15 & 16 & 17 & 18\\ 
19 & 20 & 21 & 22 & 23 & 24 & 25\\ 
26 & 27 & 28 & 29 & 30 & 31 &\\
\end{tabular}
\begin{itemize}
\item\textbf{Thu January  9} - Introduction  to Data Mining
    \begin{itemize}
        \item Begin Reading Chapter 1
    \end{itemize}
\item\textbf{Tue January 14} - Data Mining and Analysis
\item\textbf{Thu January 16} - Introduction to R
\item\textbf{Tue January 21} - Numeric Attributes
    \begin{itemize}
        \item Read Chapter 2
    \end{itemize}
\item\textbf{Thu January 23} - Numeric Attributes
    \begin{itemize}
        \item Homework 1 Assigned (Due January 30)
    \end{itemize}
\item\textbf{Tue January 28} - Categorical Attributes
    \begin{itemize}
        \item Read Chapter 3
    \end{itemize}
\item\textbf{Thu January 30} - Categorical Attributes
    \begin{itemize}
        \item Homework 2 Assigned (Due February 6)
    \end{itemize}
\end{itemize}
\hrulefill

\subsubsection*{February 2020}
\begin{tabular}{rrrrrrr}
Su & Mo & Tu & We & Th & Fr & Sa\\
   &    &    &    &    &    &  1\\ 
 2 &  3 &  4 &  5 &  6 &  7 &  8\\ 
 9 & 10 & 11 & 12 & 13 & 14 & 15\\ 
16 & 17 & 18 & 19 & 20 & 21 & 22\\ 
23 & 24 & 25 & 26 & 27 & 28 & 29\\ 
\end{tabular}
\begin{itemize}
\item\textbf{Tue February  4} - Dimensionality Reduction
    \begin{itemize}
        \item Read Chapter 7
    \end{itemize}
\item\textbf{Thu February  6} - Dimensionality Reduction
\item\textbf{Tue February 11} - Eigenpets
\item\textbf{Thu February 13} - PCA and Project 1 (Due March 10)
    \begin{itemize}
        \item Project 1 is Assigned (Due)
    \end{itemize}
\item\textbf{Tue February 18} - Itemset Mining
    \begin{itemize}
        \item Read Chapter 8
    \end{itemize}
\item\textbf{Thu February 20} - Itemset Mining
    \begin{itemize}
        \item Homework 3 is Assigned (Due February 27)
    \end{itemize}
\item\textbf{Tue February 25} - Sequence Mining
    \begin{itemize}
        \item Read Chapter 10
    \end{itemize}
\item\textbf{Thu February 27} - Sequence Mining
    \begin{itemize}
        \item Homework 4 is Assigned (Due March 5)
    \end{itemize}
\end{itemize}
\hrulefill

\subsubsection*{March 2020}
\begin{tabular}{rrrrrrr}
Su & Mo & Tu & We & Th & Fr & Sa\\
 1 &  2 &  3 &  4 &  5 &  6 &  7\\ \cline{3-3}
 8 &  9 & \multicolumn{1}{|r|}{10} & 11 & 12 & 13 & 14\\ \cline{3-3}
15 & 16 & 17 & 18 & 19 & 20 & 21\\ 
22 & 23 & 24 & 25 & 26 & 27 & 28\\
29 & 30 & 31 &    &    &    & \\
\end{tabular}
\begin{itemize}
\item\textbf{Tue March  3} - Pattern and Rule Assessment
    \begin{itemize}
        \item Read Chapter 12
    \end{itemize}
\item\textbf{Thu March  5} - Pattern and Rule Assessment
    \begin{itemize}
        \item Homework 5 is Assigned (Due March 12)
    \end{itemize}
\item\textbf{Tue March 10} - Project 1 Presentations
\item\textbf{Thu March 12} - Beginning Project 2 (Due April 2)
\item\textbf{Tue March 17} Spring Break
\item\textbf{Thu March 19} Spring Break
\item\textbf{Tue March 24} - Representative-based Clustering
    \begin{itemize}
        \item Read Chapter 13
    \end{itemize}
\item\textbf{Thu March 26} - Representative-based Clustering
    \begin{itemize}
        \item Homework 6 is Assigned (Due April 2)
    \end{itemize}
\item\textbf{Tue March 31} - Hierarchical Clustering
    \begin{itemize}
        \item Read Chapter 14
    \end{itemize}
\end{itemize}
\hrulefill

\subsubsection*{April 2020}
\begin{tabular}{rrrrrrr}
Su & Mo & Tu & We & Th & Fr & Sa\\ \cline{5-5}
   &    &    &  1 &  \multicolumn{1}{|r|}{2} &  3 &  4\\ \cline{5-5}
 5 &  6 &  7 &  8 &  9 & 10 & 11\\
12 & 13 & 14 & 15 & 16 & 17 & 18\\
19 & 20 & 21 & 22 & 23 & 24 & 25\\ 
26 & 27 & 28 & 29 & 30 &    &\\ 
\end{tabular}
\begin{itemize}
\item\textbf{Thu April  2} - Project 2 Presentations
\item\textbf{Tue April  7} - Begin Project 3 and Clustering
Validations
    \begin{itemize}
        \item Read Chapter 17
        \item Project 3 Assigned (Due April 24)
    \end{itemize}
\item\textbf{Thu April  9} - Clustering Validations
\item\textbf{Tue April 14} - Project 2 Presentations
\item\textbf{Tue April 21} - Support Vector Machines
    \begin{itemize}
        \item Read Chapter 21
    \end{itemize}
\item\textbf{Fri April 24 3:30PM} - Project 3 Presentations
\end{itemize}
\hrulefill


% Fifth Section %%%%%%%%%%%%%%%%%%%%%%%%%%%%%%%%%%%%%%%%%%%

\newpage
\section*{Course Policies}

\subsection*{Late Policy}
No late work will be accepted under any circumstances (except as mercy
and decency may dictate in extremely rare events).

\subsection*{Extra Credit}
No extra credit will be given under any circumstances.

\subsection*{Excused Absences}
In some cases, absences may be excused. These include:
\begin{itemize}
    \item School Sanctioned Events (Sports, Concerts, etc.)
    \item Severe Illness
    \item Family Emergencies
    \item Court Appearance / Jury Duty
\end{itemize}
In the case of a school event, notice must be given at least one week
prior to the absence. The notice must include a signed note from the
faculty or staff member in charge of the event. This note must be
given in physical form, electronic notes will not be accepted.
In the case of illness, a doctor's note is required. Note that
except in extreme circumstances, doctor's appointments do not qualify as a valid reason to miss
a class. Please be respectful of the other students, and schedule
appointments during your free time.

Family emergencies will require some form of proof. Where possible,
you must give advance notice of missing a class. The exception to this
would need to be fairly severe, and hopefully it will not come up.
For court appearances and/or jury duty, you must provide a copy of
your summons. You may redact any details you wish, save for the
actual date and time of your appearance. Court appearances must be
cleared at least one week in advance.

\subsection*{Making Up Excused Absences}
Should you be in a situation in which you receive an excused absence,
this in no way will extend your due dates (excepting extreme
emergencies). You must make up any test at a designated time 
prior to your excused absence. Also, homework or projects must
be submitted prior to the class period in which they are due.

\subsection*{Communication and Extra Help}
You are always welcome at office hours for help with any
questions you may have about the course. For help at other times during the day, stop by or call my office to see if I'm available. You can also contact me by email, but often I can better help you face to face and may respond with a request that
you come to see me. Note that I do not typically respond to email between 5 p.m. and 8 a.m. You may make appointments to see me at other times if your schedule does not permit you to attend my office hours.


\subsection*{Plagiarism and Cheating}
You are expected to do your own work. Never submit the work of others,
never give unauthorized assistance to others, do not use unauthorized
aids during exams, and do not ask for help from other
faculty members without the approval of your professor. Plagiarism and cheating are serious offenses that will not be
tolerated. Explanations regarding these offenses and how they are handled can be found in the MC Student Handbook at\newline
https://www.maryvillecollege.edu/academics/catalog/handbook/section-nine/.\newline
You are expected to read and understand these policies. Offenses on specific assignments, quizzes, or exams will result
in a score of 0 on the relevant assignment, and a letter of censure will be placed in your college file. Repeat offenses will
result in further disciplinary action, including the possibility of failing the course.

\subsection*{Students with Disabilities}
Any student who feels s/he may need learning or physical
accommodation(s) based on the impact of a disability should contact Services for Students with Disabilities to discuss your
specific needs. Please contact 981-8124 to coordinate reasonable accommodations for students with documented
disabilities. The Disability Services office is located in the Learning Center in the basement of Thaw Hall. Undocumented
disabilities will not be accommodated.

\end{document}

\end{document}
